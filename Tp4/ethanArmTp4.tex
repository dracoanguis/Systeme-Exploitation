\documentclass{article}

\title{Système d'Exploitation - Tp4}
\author{Ethan Arm}
\date{}

\begin{document}
    \maketitle

    \section{Composition du dossier}
        \paragraph*{}
            Le dossier est composer de deux fichiers \emph{locker.c} et un \emph{Makefile}.
            Le fichier \emph{locker.c} contient l'intégralité de l'implémentation du programme.
    
    \section{Fonctionnement et utilisation}
        \paragraph*{}
            Pour utiliser le programme, il est néccessaire de l'installer à l'aide de la commande \emph{make}.
            Une fois installer il suffit de l'appeler avec la commande \emph{./locker file} où \emph{file} est un fichier sur lequel on veut expérimenter pour placer nos locks.
            Une fois le programme lancer une aide est affichée sur comment utiliser les commandes. Une particularité de ce programme est que l'on as uniquement besoin de lancer une instance du programme pour voir les locks.
            En effet, une commande est incluse pour crée un unique processus enfant dans lequel on peut voir les locks du processus parent.

\end{document}