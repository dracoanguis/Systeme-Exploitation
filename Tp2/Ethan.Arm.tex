\documentclass{article}    
\usepackage{minted}

\title{Système d'Exploitation - Tp2}
\author{Ethan Arm}

\begin{document}
    \maketitle

    \section{Cryptographie}
        \subsection{Concepts}
            \paragraph*{}
                On met la phrase \emph{Le manuel disait: Néccessite Windows 7 ou mieux. J'ai donc installé Linux} dans le fichier \mintinline{bash}{exercice/texte.txt} et compare le hash de ce fichier avec le hash de la phrase avec \mintinline{bash}{echo} et on trouve la même chose.
                Je suppose qu'il est possible que l'on trouve des choses différente dans le cas on l'éditeur de texte rajoute des caractère non-imprimable ou dans le cas rajoute un retour à la ligne.
                Il semblerait que mon éditeur de text (nano) et echo rajoute tout les deux un $\backslash$n et c'est pourquoi le hash est le même.

        \subsection{La librairie openssl}
            \paragraph*{}
                On compile l'exemple avec la phrase \emph{Le manuel disait: Néccessite Windows 7 ou mieux. J'ai donc installé Linux $\backslash$n} et on remarque que l'on obtient le même digest que dans le point précédent.
        
    \section{Gestion des paramètre d'un programme}
        \paragraph*{}
            On essaye d'utiliser la fonctiond'exemple de \mintinline{c}{getopt()} et cela fonctionne.
    
    \section{Intégration: le programme à réaliser}
        

\end{document}