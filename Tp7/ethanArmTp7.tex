\documentclass{article}

\author{Ethan Arm}
\date{}
\title{Système d'exploitation -Tp7}

\begin{document}
    \maketitle

    \section{Composition}
        \paragraph*{}
            Le dossier contient 3 fichier, un \emph{Makefile} a utiliser pour compiler le code.
            Le fichier \emph{cuisinier.c} et le fichier \emph{serveur.c} qui contienent le code du Tp.

    \section{Installation}
        \paragraph*{}
            Pour installer le program il faut lancer la commande \emph{make}, puis la commande \emph{make serveur}.
    
    \section{Utilisation}
        \paragraph*{}
            Lors de l'utilisation normale, il suffit de lancer le programme \emph{cuisinier}, 
            puis dans un autre pseudo-terminal lancer le programme \emph{serveur}.
            On voit alors tout le processus se dérouler accompagner de commentaire.
    
    \section{Erreur}
        \paragraph*{}
            En cas d'interruption forcé du programme les fichiers utilisé pour la méoire partagé ne sont pas effacé naturelement,
            heureusement il suffit de lancer la commande \emph{make reset} pour les effacé automatiquement et reglé le problème.

\end{document}