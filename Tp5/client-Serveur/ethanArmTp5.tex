\documentclass{article}

\title{Système d'Exploitation - Tp5}
\author{Ethan Arm}
\date{}

\begin{document}
    \maketitle

    \section{Composition}
        \paragraph*{}
            Le dossier contient 3 fichier, un \emph{Makefile} a utiliser pour compiler le code, 
            le fichier \emph{client.c} qui est l'implémentation du côté client et un fichier \emph{serveur.c} qui est le côté serveur.

    \section{Installation}
        \paragraph*{}
            L'installation se passe en deux étapes, puisque nous avons deux programmes, la première est simplement de taper la commande \emph{make}.
            Celle-ci installera la côté serveur et la deuxième est de taper \emph{make client} qui installera la partie client.

    \section{Utilisation}
        \paragraph*{}
            Lors d'une utilisation normal le serveur est d'abord lancé avec un port valide, et ensuite un ou plusieurs clients sont lancé avec l'adresse Ip du serveur et le port plus tôt sélectionné.
            On as alors les messages de conexion de chaque client qui s'affichent sur le terminal serveur, par ailleur une couleur est associé à chaque connection.
            On joue ensuite côté client l'objectif étant de deviner le nombre aléatoirement choisie par le servuer entre 0 et 255 (1 octet, pour respecter la consigne des signaux d'envoie).
            Une fois que la partie est terminé la connexion est terminé et le processus client s'arrête.

\end{document}