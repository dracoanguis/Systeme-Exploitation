\documentclass{article}    
\usepackage{minted}

\newcommand{\bash}[1]{\mintinline{bash}{#1}}

\title{Système d'Exploitation - Tp2}
\author{Ethan Arm}

\begin{document}
    \maketitle

    \section{Cryptographie}
        \subsection{Concepts}
            \paragraph*{}
                On met la phrase \emph{Le manuel disait: Néccessite Windows 7 ou mieux. J'ai donc installé Linux} dans le fichier \mintinline{bash}{exercice/texte.txt} et compare le hash de ce fichier avec le hash de la phrase avec \mintinline{bash}{echo} et on trouve la même chose.
                Je suppose qu'il est possible que l'on trouve des choses différente dans le cas on l'éditeur de texte rajoute des caractère non-imprimable ou dans le cas rajoute un retour à la ligne.
                Il semblerait que mon éditeur de text (nano) et echo rajoute tout les deux un $\backslash$n et c'est pourquoi le hash est le même.

        \subsection{La librairie openssl}
            \paragraph*{}
                On compile l'exemple avec la phrase \emph{Le manuel disait: Néccessite Windows 7 ou mieux. J'ai donc installé Linux $\backslash$n} et on remarque que l'on obtient le même digest que dans le point précédent.
        
    \section{Gestion des paramètre d'un programme}
        \paragraph*{}
            On essaye d'utiliser la fonctiond'exemple de \mintinline{c}{getopt()} et cela fonctionne.
    
    \section{Intégration: le programme à réaliser}
        \paragraph*{}
            On résalise le programme \mintinline{bash}{hash.exe}, celui-ci est réparti en neuf fichiers.
            Deux sont \mintinline{bash}{liste.c} et \mintinline{bash}{liste.h} quis ont une librairie pour utilisé des listes chaînées que j'ai écrite au préalable.
            Deux autres sont \mintinline{bash}{crypt.c} et \bash{crypt.h}, ils contienne toutes les fonction néccessaire a hashé les fichiers et les phrases.
            Malgré leur noms les fichiers \bash{option.c} et \bash{option.h} ne gère pas exactement les options en effet celles-ci sont gérer dans \bash{hash.c}, une certaine erreur de design mais cela semblais faire plus sense que toute la gestion de la fonction \mintinline{c}{getopt()} se fasse dans la fonction \mintinline{c}{main()}.
            Ainsi ces deux fichiers gèrent certes une partie mineur mais quand même une partie des argument donné par les options. 
            Et enfin le fichier \bash{hash.c} contient la fonction \mintinline{c}{main()} qui est l'entrée de notre programme.
        
        \paragraph*{}
            Ils existent encore d'autre fichiers dans notre projet qui sont le \bash{Makefile} et \bash{test.sh}, le premier contient les commande de compilation et le deuxième est un petit scripte pour montrer l'utilisation de notre programme et diverse de ces comportement.
            Notament on as rajouté l'option \emph{-h} qui affiche une aide pour notre programme.

\end{document}