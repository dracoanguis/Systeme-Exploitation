\documentclass{article}

\title{Système d'Exploitation - Tp3}
\author{Ethan Arm}
\date{}

\begin{document}
    \maketitle

    \section{Composition du dossier}
        \paragraph*{}
            On réalise l'implémentation du programme \emph{ultra-cp}, celui-ci se décompose en deux module et un fichier principal.
            Le module \emph{read} permet la lecture récursive des fichiers et le module \emph{copy} permet leur copie récursive.
            Nous avons de plus un \emph{Makefile} afin de compiler notre programme. Et enfin il y a le fichier \emph{ultra-cp.c} qui contient notre programme.

    \section{Fonctionnement et utilisation}
        \paragraph*{}
            Pour utiliser le programme il faut d'abord l'installer grace a la commande \emph{make}, il est alors installer dans le dossier.
            Une aide sur le programme est disponible si on passe l'option \emph{-h} ou si on ne donne aucun paramètre.
            Ce programme affiche un listing si un unique fichier ou dossier lui est passé en paramètre. Il effectue une copie d'un dossier vers l'autre si deux lui sont passé en paramètres.
            Et enfin il effectue une copie récursive de fichier et dossier si plus de trois paramètres lui sont donné. Attention: dans ce cas il est néccessaire que le dernier paramètre soit le dossier de destination.
            Enfin ce programme prend trois options \emph{h,f,a} comme dis plus haut \emph{-h} affiche une aide, \emph{-a} modifie les permissions des fichiers dans la destination s'il existe dans la source, même s'ils n'ont pas été modifié. et dernièrement l'option \emph{-f} permet la copie de lien symbolique dans la destination contrairement à la normale ou le fichier pointé est ignoré (malheuresement ces deux dernières ne sont pas implémenté ).

\end{document}